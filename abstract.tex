A central problem in neuroscience is to understand how memories are formed as a result of the activities of neurons. Valiant’s neuroidal model attempted to address this question by modeling the brain as a random graph and memories as subgraphs within that graph. However the question of memory capacity within that model has not been explored: how many memories can the brain hold? Valiant introduced the concept of interference between memories as the defining factor for capacity; excessive interference signals the model has reached capacity. Since then, exploration of capacity has been limited, but recent investigations have delved into the capacity of the Assembly Calculus, a derivative of Valiant's Neuroidal model. In this paper, we provide rigorous definitions for capacity and interference and present theoretical formulations for the. memory capacity within a finite set, where subsets represent memories. We propose that these results can be adapted to suit both the Neuroidal model and Assembly calculus. Furthermore, we substantiate our claims by providing simulations that validate the theoretical findings. Our study aims to contribute essential insights into the understanding of memory capacity in complex cognitive models, offering potential ideas for applications and extensions to contemporary models of cognition.