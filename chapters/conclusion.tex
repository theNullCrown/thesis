\chapter{Conclusion}

Inspired by advances in modern computational neuroscience, we rigorously defined and studied the notion of ``capacity'' and ``interference'' in a set both theoretically and emprically. We also provided ideas to extend these results to more structured objects like graphs with more advanced algorithms for adding subobjects to the universe.

\section{Future work}

Here we discuss some potential future work building off this study:

\begin{itemize}
    \item Adapt lemma \ref{lemma:k-int-prob} to find the expected interference in the case of other memory creation algorithms. The rest of the theorems will follow similarly to be able to find the capacity of the model. 
    \item Instead of bounding the subset sizes, assume the subset sizes are drawn from a distribution with a given mean $r$ and find the expected subset capacity. This will involve finding the expectation of the hypergeometric PMF as a function of two random variables. 
\end{itemize}

\section{Closing thoughts}

We believe this study will inspire more computational neuroscientists to tackle the intriguing question of capacity as they develop models that slowly and steadily demistify the human brain. 
