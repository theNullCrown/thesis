Modern neuroscience has embraced random graph-based models, notably popularized by Valiant's seminal 2005 paper, to comprehend cognition. Memories are construed as subgraphs within these models, yet the issue of memory capacity, initially addressed by Valiant, remains unexplored. Valiant introduced the concept of interference between memories as the defining factor for capacity; excessive interference signals the model has reached capacity. Since then, exploration of capacity has been limited, but recent investigations have delved into the capacity of the Assembly calculus, a derivative of Valiant's Neuroidal model. In this paper, we provide rigorous definitions for capacity and interference and present theoretical formulations for memory capacity within a finite set, where subsets represent memories. We propose that these results can be adapted to suit the Neuroidal model and Assembly calculus. Furthermore, we substantiate our claims by providing simulations that validate the theoretical findings. Our study aims to contribute essential insights into the understanding of memory capacity in complex cognitive models, offering potential ideas for applications and extensions to contemporary models of cognition. 
